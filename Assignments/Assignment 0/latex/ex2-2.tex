% !TeX root = ex2.tex
\subsection{Capture Images through Webcam}

\paragraph{Problem}
Use a webcam, capture images and save them to a folder

\paragraph{Experiments \& Learning} Experiments performed are listed below

\begin{enumerate}
    \item Learned to capture images from a webcam using \texttt{VideoCapture}
\end{enumerate}

\paragraph{Solution}
Run the following command

\begin{verbatim}
    python .\webcam_capture.py -c 0 -o "./camimgs"
\end{verbatim}

The code is included in listing \ref{lst:q2-webcam-cap}. Output is shown in figure \ref{fig:q2-webcam-imgs}.

\lstinputlisting[language=python, caption={webcam\_capture.py}, label=lst:q2-webcam-cap]{./../python/webcam_capture.py}

Four images were captured 

\begin{verbatim}
    Capturing image to './camimgs/1.jpg'
    Capturing image to './camimgs/2.jpg'
    Capturing image to './camimgs/3.jpg'
    Capturing image to './camimgs/4.jpg'
    Quit command received
\end{verbatim}

\begin{figure}[t]
    \centering
    \begin{subfigure}[b]{0.45\textwidth}
        \includegraphics[width=\textwidth]{1.jpg}
        \caption{Image 1}
    \end{subfigure}
    \begin{subfigure}[b]{0.45\textwidth}
        \includegraphics[width=\textwidth]{2.jpg}
        \caption{Image 2}
    \end{subfigure}
    \begin{subfigure}[b]{0.45\textwidth}
        \includegraphics[width=\textwidth]{3.jpg}
        \caption{Image 3}
    \end{subfigure}
    \begin{subfigure}[b]{0.45\textwidth}
        \includegraphics[width=\textwidth]{4.jpg}
        \caption{Image 4}
    \end{subfigure}
    \caption{Images captured through webcam}
    \label{fig:q2-webcam-imgs}
    \small
        Four images captured through a webcam
\end{figure}
