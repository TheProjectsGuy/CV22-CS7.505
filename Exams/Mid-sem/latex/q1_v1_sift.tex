% !TeX root = q1_v1.tex

\subsection{SIFT}

Scale Invariant Feature Transform (SIFT) is an image feature generation method introduced by David G. Lowe in \cite{sift-original-lowe}. A more explained iteration was presented in \cite{sift-detailed-lowe} with some revisions to the model. The primary contribution of the author was exploring the features in multiple scales (through an image pyramid), which makes the keypoint descriptors \emph{scale} invariant.

\paragraph*{Detector}

The keypoint detector has the following basic steps

\begin{enumerate}
    \item Construction of DoG (Difference of Gaussian) image pyramid: The input image resolution is increased (scaled up) by a factor of two using bilinear interpolation.
    Then two successive gaussian blurs are applied, yielding image \texttt{A} (less blurred) and \texttt{B} (more blurred).
    Subtracting image \texttt{B} from \texttt{A} given the Difference of Gaussian image.
    This is repeated for scales of $1.5$ in each direction (up and down). This scaling factor was later revised to $2$.

    \item Achieve keypoint locations: The local extrema in the DoG image is a keypoint. 
    First, eight neighbor comparisons are made at the same scale. 
    Then, if the point is in extrema (maximum or minimum), comparisons are made at higher scales (position interpolation to maintain scale).
    
    \item Extract Keypoint orientations: The image \texttt{A} is used to compute the gradient magnitudes and orientations. The magnitudes are thresholded to $0.1$ times the maximum gradient value (to reduce illumination effects).
    A histogram of gradient orientations in the local neighborhood of keypoints is created. The weight of the orientations is the thresholded gradient values.
    This histogram (containing $36$ bins covering $0^\circ$ to $360^\circ$) is smoothened, and the peak is chosen as the gradient orientation.
\end{enumerate}

In the end, the keypoint locations (on the image) and orientations are obtained. The direction is used to achieve rotation-invariant features.

\paragraph*{Descriptor}

The keypoint descriptor (as presented in the original work in \cite{sift-original-lowe}) has the following basic steps

\begin{enumerate}
    \item Reorient the local region around the keypoint. This is basically to set the local orientation (of keypoint) as a reference. This is done by simple subtraction of gradient orientations in later steps.
    \item Subdivide the local region: The local region (within the radius of $8$ pixels) is sub-divided into a $4 \times 4$ sub-array, with each sub-array having an 8-bin gradient histogram.
    \item Run the same on a larger scale version: On one scale higher in the pyramid, perform the above step but with a $2 \times 2$ sub-array (still $8$ bins in the histogram).
\end{enumerate}

Note that the gradient directions for the histogram are not just the gradients at the center pixel but are interpolated in the $n \times n$ grid (sub-array with $n=4 $ or $2$). The total number of SIFT descriptors (length) for a keypoint is $8 \times 4 \times 4 + 8 \times 2 \times 2 = 160$. 

In the revised edition \cite{sift-detailed-lowe}, a $16 \times 16$ local region is sub-divided into $4\times 4$ grid, with each grid having $8$ orientation bins (from histogram). Therefore, the new descriptor length becomes $8 \times 4 \times 4 = 128$. It is found that this is much faster to compute and doesn't have a large compromise on performance.
