% !TeX root = main.tex

\section{Questions}

I formulated the following questions for the exam.

% --------- Question texts: Begin ---------
% Question 1
\newcommand{\questionOne}{
    % Contents of question 1
Enunciate any two methods of keypoint detection and description. Can these be used in an ensemble? If yes, then how?

}

% Question 2
\newcommand{\questionTwo}{
    % Contents of question 2
Describe and derive the Fundamental Matrix, Essential Matrix, and Homography Matrix (for the case of pure rotation). When there are many correspondences between two images, can these methods be used to filter out the best correspondences?

}
% ---------- Question texts: End ----------

\begin{enumerate}
    \item \questionOne
    \item \questionTwo
\end{enumerate}

\subsection{Reasons}

The reasons why I feel these questions are worthy and/or interesting

\subsubsection*{Q1: Keypoint detection and description}

\paragraph*{Question}
\begin{displayquote}
    \questionOne
\end{displayquote}

\paragraph*{Reason}
This question promotes a deeper dive into the reading material for the theoretical methods taught in class. It should serve as a quick and good reference for traditional keypoint detection and description methods.
The aim is to have a good information archive, created through reading the original text, well summarized, in one place.

Ensemble techniques have recently caught steam, especially in the age of deep learning. Exploring such options for traditional methods could yield stronger baselines for traditional feature detection and description methods.


\subsection*{Q2: Relating matrices and RANSAC}

\paragraph*{Question}
\begin{displayquote}
    \questionTwo
\end{displayquote}

\paragraph*{Reason}
This question is to lay a foundation for matrices relating to two images. Though the derivation is not traditionally important, it is good to have the theoretical backbone in one place. Random Sample Consensus (RANSAC) is a very popular method to boost the performance of a correspondence matching algorithm. This question aims to derive the theory behind it and also give a direction on how it can be applied using the knowledge of these basic matrices in computer vision. It concludes with examples and references to a python package that can perform RANSAC using the matrices.
