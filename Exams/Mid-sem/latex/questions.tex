% !TeX root = main.tex

\section{Questions}

I formulated the following questions for the exam.

% --------- Question texts: Begin ---------
% Question 1
\newcommand{\questionOne}{
    % Contents of question 1
Enunciate any two methods of keypoint detection and description. Can these be used in an ensemble? If yes, then how?

}

% Question 2
\newcommand{\questionTwo}{
    % Contents of question 2
Describe and derive the Fundamental Matrix, Essential Matrix, and Homography Matrix (for the case of pure rotation). When there are many correspondences between two images, can these methods be used to filter out the best correspondences?

}

% Question 3
\newcommand{\questionThree}{
    % Contents of question 3
You've been provided with an image taken from a self-driving car that shows another car in front. A camera has been placed on top of the car, 1.65 m from the ground. The camera intrinsic matrix $K$ is provided.
Your task is to draw a 3D-bounding box around the car in front. Your approach should be to place eight points in the 3D world such that they surround all the corners of the car, then project them onto the image and connect the projected image points using lines. Make a python program for this.

Assume that the image plane is perfectly perpendicular to the ground.
You might have to apply a small 5° rotation about the vertical axis to align the box perfectly. 
Rough car dimensions - h: 1.38 m, w: 1.51, l: 4.10.
Also, estimate the approximate translation vector to the mid-point of the two rear wheels of the car in the camera frame.

}

% ---------- Question texts: End ----------

\begin{enumerate}
    \item \questionOne
    \item \questionTwo
    \item \questionThree
\end{enumerate}

\subsection{Reasons}

The reasons why I feel these questions are worthy and interesting

\begin{itemize}
    \item The set has the perfect balance of theory, math, and critical thinking
    \begin{itemize}
        \item Question 1 is theoretical and involves reading papers.
        \item Question 2 is towards practical mathematics.
        \item Question 3 involves programming.
    \end{itemize}
    \item The questions can have concrete answers and are not vague.
\end{itemize}

\subsubsection*{Q1: Keypoint detection and description}

\paragraph*{Question}
\begin{displayquote}
    \questionOne
\end{displayquote}

\paragraph*{Reason}
This question promotes a deeper dive into the reading material for the theoretical methods taught in class. It should serve as a quick and good reference for traditional keypoint detection and description methods.
The aim is to have a good information archive, created through reading the original text, well summarized, in one place.

Ensemble techniques have recently caught steam, especially in the age of deep learning. Exploring such options for traditional methods could yield stronger baselines for traditional feature detection and description methods.


\subsection*{Q2: Relating matrices and RANSAC}

\paragraph*{Question}
\begin{displayquote}
    \questionTwo
\end{displayquote}

\paragraph*{Reason}
This question is to lay a foundation for matrices relating to two images. Though the derivation is not traditionally important, it is good to have the theoretical backbone in one place. Random Sample Consensus (RANSAC) is a very popular method to boost the performance of a correspondence matching algorithm. This question aims to derive the theory behind it and also give a direction on how it can be applied using the knowledge of these basic matrices in computer vision. It concludes with examples and references to a python package that can perform RANSAC using the matrices.

\subsection*{Q3: Bounding Box}

\paragraph*{Question}
\begin{displayquote}
    \questionThree
\end{displayquote}

\paragraph*{Reason}
The question is to implement the camera model and transformations in Python to solve a real-world problem. The question can test the understanding of the camera model if solved correctly. Plus, it will be something that can be extended and is the only interactive part of this exam.
