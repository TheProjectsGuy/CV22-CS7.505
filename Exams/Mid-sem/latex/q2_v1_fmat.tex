% !TeX root = q2_v1.tex
% About the fundamental matrix

\subsection{Fundamental Matrix}

Say we have three vectors $\vec{a}$, $\vec{b}$, and $\vec{c}$. Their triple product $\left \langle \vec{a} \;\; \vec{b} \;\; \vec{c} \right \rangle = \vec{a} \cdot \left ( \vec{b} \times \vec{c} \right )$ is the volume of the parallelepiped formed by the three vectors.

Since the three vectors $\overrightarrow{O_1 X}$, $\overrightarrow{O_1 O_2}$, and $\overrightarrow{O_2 X}$, all lie on the same plane, their triple product will be zero. That is  $\left \langle O_1 X \;\; O_1 O_2 \;\; O_2 X \right \rangle = \mathbf{0}$.

From camera projection properties, we can write

\begin{align}
    x' = \mathbf{K}' \mathbf{R}' \left [ \mathbf{I} \mid -\mathbf{X}_{O'} \right ] X
    &&
    x'' = \mathbf{K}'' \mathbf{R}'' \left [ \mathbf{I} \mid -\mathbf{X}_{O''} \right ] X
    \label{eq:q2-camera-projection}
\end{align}

Where $\mathbf{K}'$, $\mathbf{R}' \left [ \mathbf{I} \mid -\mathbf{X}_{O'} \right ]$ and $\mathbf{K}''$, $\mathbf{R}'' \left [ \mathbf{I} \mid -\mathbf{X}_{O''} \right ]$ are camera intrinsic and extrinsic parameters (for camera 1 and camera 2) respectively. Note that all the above terms are in \textit{homogeneous coordinates}. The vector $X$ can be assumed to be unit-scale (last term - the scaling factor - is 1).

We know that $\overrightarrow{O_1 X} = X - X_{O'} \equiv \mathbf{R}'^{-1} \mathbf{K}'^{-1} x'$ and $\overrightarrow{O_2 X} = X - X_{O''} \equiv \mathbf{R}''^{-1} \mathbf{K}''^{-1} x''$. Another reduction is $\overrightarrow{O_1 O_2} = b$ for the baseline vector (joining the two camera centers).

Therefore, the triple product constraint mentioned above can be reduced to

\begin{align}
    \left \langle O_1 X \;\; O_1 O_2 \;\; O_2 X \right \rangle = \mathbf{0}
    \Rightarrow \left ( X - X_{O'} \right ) \cdot \left ( b \times \left ( X - X_{O''} \right ) \right ) 
    \equiv \left ( \mathbf{R}'^{-1} \mathbf{K}'^{-1} x' \right ) \cdot \left ( b \times \left ( \mathbf{R}''^{-1} \mathbf{K}''^{-1} x'' \right ) \right ) = 0
    \nonumber
\end{align}

Using $a \cdot b = a^\top b$ and $a \times b = \left [ a \right ]_\times b$ (where $\left [ a \right ]_\times$ is the cross product skew symmetric matrix), we can reduce the above equation to

\begin{align}
    \left \langle O_1 X \;\; O_1 O_2 \;\; O_2 X \right \rangle =& \left ( \mathbf{R}'^{-1} \mathbf{K}'^{-1} x' \right ) \cdot \left ( b \times \left ( \mathbf{R}''^{-1} \mathbf{K}''^{-1} x'' \right ) \right ) =
        \left ( \mathbf{R}'^{-1} \mathbf{K}'^{-1} x' \right )^\top \left [ b \right ]_\times \left ( \mathbf{R}''^{-1} \mathbf{K}''^{-1} x'' \right )
    \nonumber \\
    =& x'^\top \left ( \mathbf{K}'^{-\top} \mathbf{R}'^{-\top} \left [ b \right ]_\times \mathbf{R}''^{-1} \mathbf{K}''^{-1} \right ) x'' = x'^\top \mathbf{F} x'' = 0
    \label{eq:q2-fmat-eq}
\end{align}

The equation \ref{eq:q2-fmat-eq} is the basis for two points (in different images) to be projected to the same point in the 3D world. If the points correspond in the 3D world, they must satisfy the equation. However, the converse is not necessarily valid, as we will see later. Let us get the intuition of the epipolar line and the epipoles through the fundamental matrix.

\subsubsection*{Epipolar line}

Say we have a point $x'$ in one image, and we want to find the corresponding point $x''$ in a second image. Assume that we have $\mathbf{F}$ (the fundamental matrix) relating the two images.

Referring to figure \ref{fig:q2-epipolar-geometry}, our job would become much easier if we know the \textit{epipolar line} of $x'$ in the second image (the line $e''x''$). Let us call this line $l''$ (since it's in the second image).

For a true $x''$ to lie on $l''$, it must satisfy $x'' \cdot l'' = x''^\top l'' = l''^\top x'' = 0$. From equation \ref{eq:q2-fmat-eq}, we know that $x' \mathbf{F} x'' = 0$.

Matching the two results, we get $l''^\top = x' \mathbf{F} \Rightarrow l'' = \mathbf{F}^\top x' $ as the equation of the epipolar line in the second image (of the point $x'$ in the first image). Now, a search along this line in the second image has higher chances of yielding the true $x''$.

\subsubsection*{Epipoles}

We know that the epipolar line in the second image (of a point $x'$ in the first image) is given by $l'' = \mathbf{F}^\top x'$.

We know that the epipole $e''$ (in the second image) is the projection of $O_1$ in the second image. That is $e'' = \mathbf{P}'' X_{O''}$ (where $\mathbf{P}'' = \mathbf{K}'' \mathbf{R}'' \left [ \mathbf{I} \mid -\mathbf{X}_{O''} \right ]$ is the second camera's projection matrix). We also know that the epipole $e''$ lies on line $l''$, since all epipolar lines intersect at the epipoles (this is seen by considering another epipolar plane with a 3D point $Y$ in figure \ref{fig:q2-epipolar-geometry}). We therefore have $l''^\top e'' = 0$.

For any point $x'$ in the first image, there will be a unique epipolar line $l''$ in the second image. We therefore have

\begin{equation}
    l''^\top e'' = \left ( \mathbf{F}^\top x' \right )^\top e'' = x'^\top \mathbf{F} e'' = \left ( \mathbf{F} e'' \right )^\top x' = 0
    \label{eq:q2-epipole-fmat-eq}
\end{equation}

We have two conditions: $x'$ can be any valid point in image 1 (in homogeneous coordinates) \textit{and} equation \ref{eq:q2-epipole-fmat-eq} always has to hold true. The only possibility where both these conditions hold true is when $\left ( \mathbf{F} e'' \right )^\top = \mathbf{0}^\top$ (it is a row of three zeros). We therefore have $\mathbf{F} e'' = \mathbf{0}$.

In other words, the epipole $e''$ is the null space of the fundamental matrix $\mathbf{F}$. We can obtain the epipoles from $\mathbf{F}$ through eigendecomposition (by obtaining the eigenvector with the least - ideally zero - eigenvalue).

\subsubsection*{Transpose Relation}

Equation \ref{eq:q2-fmat-eq} gives the fundamental matrix relating image 1 to image 2 (simply because point $x'$ in image 1 comes before point $x''$ which is in image 2). Transposing it gives

\begin{equation}
    (x'^\top)_{1,3} \, \mathbf{F}_{3,3} \, (x'')_{3, 1} = 0 \Rightarrow \left ( x'^\top \mathbf{F} x'' \right )^\top = 0
    \Rightarrow x''^\top \mathbf{F}^\top x' = 0
    \label{eq:q2-fmat-transpose-rel}
\end{equation}

Therefore, the fundamental matrix relating the second image to the first is given by the \textit{transpose}. That is, if $^1_2\mathbf{F} = \mathbf{F}$, then $^2_1\mathbf{F} = \mathbf{F}^\top$.
